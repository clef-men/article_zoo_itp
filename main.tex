\documentclass[a4paper,anonymous,cleveref]{lipics-v2021}

% ------------------------------------------------

\bibliographystyle{plainurl}

% ------------------------------------------------

\usepackage{minted}

\DeclareUnicodeCharacter{3B9}{$\iota$}
\DeclareUnicodeCharacter{2200}{$\forall$}
\DeclareUnicodeCharacter{2203}{$\exists$}
\DeclareUnicodeCharacter{2260}{$\neq$}
\DeclareUnicodeCharacter{3A3}{$\Sigma$}
\DeclareUnicodeCharacter{2217}{$\ast$}
\DeclareUnicodeCharacter{231C}{$\ulcorner$}
\DeclareUnicodeCharacter{231D}{$\urcorner$}

% ------------------------------------------------

\usepackage{mathtools}

% ------------------------------------------------

\usepackage{macros}

\usepackage{mycomments}

% ------------------------------------------------

\title{
  \Zoo: A framework for the verification of concurrent \OCaml~5 programs using separation logic
}

\author{Clément Allain}{INRIA Paris, France}{clement.allain@inria.fr}{https://orcid.org/0009-0005-2972-5181}{}

\author{Gabriel Scherer}{INRIA Paris, France}{gabriel.scherer@inria.fr}{https://orcid.org/0000-0003-1758-3938}{}

\authorrunning{C. Allain and G. Scherer}

\Copyright{Clément Allain and Gabriel Scherer}

\ccsdesc[100]{\textcolor{red}{Replace ccsdesc macro with valid one}} %TODO mandatory: Please choose ACM 2012 classifications from https://dl.acm.org/ccs/ccs_flat.cfm 

\keywords{\Rocq, program verification, separation logic}

% ------------------------------------------------

%Editor-only macros:: begin (do not touch as author)
\EventEditors{John Q. Open and Joan R. Access}
\EventNoEds{2}
\EventLongTitle{16th International Conference on Interactive Theorem Proving (ITP 2025)}
\EventShortTitle{ITP 2025}
\EventAcronym{ITP}
\EventYear{2025}
\EventDate{September 27 -- October 03, 2025}
\EventLocation{Reykjavik, Iceland}
\EventLogo{}
\SeriesVolume{42}
\ArticleNo{23}

% ------------------------------------------------
% ------------------------------------------------

\begin{document}

% ------------------------------------------------

\maketitle

\begin{abstract}
  The release of \OCaml~5, which introduced parallelism into the language, drove the need for safe and efficient concurrent data structures.
New libraries like \Saturn~\cite{saturn} aim at addressing this need.
From the perspective of formal verification, this is an opportunity to apply and further state-of-the-art techniques to provide stronger guarantees.

We present a framework for verifying fine-grained concurrent \OCaml~5 algorithms.
Following a pragmatic approach, we support a limited but sufficient fragment of the language whose semantics has been carefully formalized to faithfully express such algorithms.
Source programs are translated to a deeply-embedded language living inside \Rocq where they can be specified and verified using the \Iris~\cite{DBLP:journals/jfp/JungKJBBD18} concurrent separation logic.
\end{abstract}

\section{Introduction}
\label{sec:introduction}

Designing concurrent algorithms, in particular \href{https://en.wikipedia.org/wiki/Non-blocking_algorithm#Lock-freedom}{\emph{lock-free}} algorithms, is a notoriously difficult task.
In this paper, we are concerned with proving the correctness of these algorithms.

\begin{figure}[htb]
\begin{minted}{ocaml}
type 'a t =
  'a list Atomic.t

let create () =
  Atomic.make []

let rec push t v =
  let old = Atomic.get t in
  let new_ = v :: old in
  if not @@ Atomic.compare_and_set t old new_ then (
    Domain.cpu_relax () ;
    push t v
  )

let rec pop t =
  match Atomic.get t with
  | [] ->
      None
  | v :: new_ as old ->
      if Atomic.compare_and_set t old new_ then (
        Some v
      ) else (
        Domain.cpu_relax () ;
        pop t
      )
\end{minted}
\caption{Implementation of a concurrent stack}
\label{fig:stack}
\end{figure}

\paragraph{Example 1: physical equality.}

Consider, for example, the \OCaml implementation of a concurrent stack~\cite{thomas1986systems} in \cref{fig:stack}.
Essentially, it consists of an atomic reference to a list that is updated atomically using the \mintinline{ocaml}{Atomic.compare_and_set} primitive.
While this simple implementation---it is indeed one of the simplest lockfree algorithms---may seem easy to verify, it is actually more subtle than it looks.

Indeed, the semantics of \mintinline{ocaml}{Atomic.compare_and_set} involves \emph{physical equality}: if the content of the atomic reference is physically equal to the expected value, it is atomically updated to the new value.
Comparing physical equality is tricky and can be dangerous---this is why \emph{structural equality} is often preferred---because the programmer has few guarantees about the \emph{physical identity} of a value.
In particular, the physical identity of a list, or more generally of an inhabitant of an algebraic data type, is not really specified.
The only guarantee is: if two values are physically equal, they are also structurally equal.
Apparently, we don't learn anything interesting when two values are physically distinct.
Going back to our example, this is fortunately not an issue, since we always retry the operation when \mintinline{ocaml}{Atomic.compare_and_set} returns \mintinline{ocaml}{false}.

Looking at the standard runtime representation of \OCaml values, this makes sense.
The empty list is represented by a constant while a non-empty list is represented by pointer to a tagged memory block.
Physical equality for non-empty lists is just pointer comparison.
It is clear that two pointers being distinct does not imply the pointed memory blocks are.

From the viewpoint of formal verification, this means we have to carefully design the semantics of the language to be able to reason about physical equality and other subtleties of concurrent programs.
Essentially, the conclusion we can draw is that the semantics of physical equality and therefore \mintinline{ocaml}{Atomic.compare_and_set} is non-deterministic: we cannot determine the result of physical comparison just by looking at the abstract values.

\paragraph{Example 2: when physical identity matters.}

Consider another example given in \cref{fig:rcfd}: the \mintinline{ocaml}{Rcfd.close}\footnote{\url{https://github.com/ocaml-multicore/eio/blob/main/lib_eio/unix/rcfd.ml}} function from the \Eio~\cite{eio} library.
Essentially, it consists in protecting a file descriptor using reference counting.
Similarly, it relies on atomically updating the \mintinline{ocaml}{state} field using \mintinline{ocaml}{Atomic.Loc.compare_and_set}\footnote{Here, we make use of atomic record fields that were \href{https://github.com/ocaml/ocaml/pull/13404}{recently introduced} in \OCaml.}.
However, there is a complication.
Indeed, we claim that the correctness of \mintinline{ocaml}{close} derives from the fact that the \mintinline{ocaml}{Open} state does not change throughout the lifetime of the data structure; it can be replaced by a \mintinline{ocaml}{Closing} state but never by another \mintinline{ocaml}{Open}.
In other words, we want to say that 1) this \mintinline{ocaml}{Open} is \emph{physically unique} and 2) \mintinline{ocaml}{Atomic.Loc.compare_and_set} therefore detects whether the data structure has flipped into the \mintinline{ocaml}{Closing} state.
In fact, this kind of property appears frequently in lockfree algorithms; it also occurs in the \Kcas~\cite{kcas} library\footnote{\url{https://github.com/ocaml-multicore/kcas/blob/main/doc/gkmz-with-read-only-cmp-ops.md}}.

Once again, this argument requires special care in the semantics of physical equality.
In short, we have to reveal something about the physical identity of some abstract values.
Yet, we cannot reveal too much---in particular, we cannot simply convert an abstract value to a concrete one (a memory location)---, since the \OCaml compiler performs optimizations like sharing of immutable constants, and the semantics should remain compatible with adding other optimizations later on, such as forms of hash-consing.

\paragraph{A formalized \OCaml fragment for the verification of concurrent algorithms.}

These subtle aspects, illustrated through two realistic examples, justify the need for a faithful formal semantics of a fragment of \OCaml tailored for the verification of concurrent algorithms.
Ideally, of course, this fragment would include most of the language.
However, the direct practical aim of this work---the verification of real-life libraries like \Saturn~\cite{saturn}---led us to the following design philosophy: only include what is actually needed to express and reason about concurrent algorithms in a convenient way.

In this paper, we show how we have designed a practical framework, \Zoo\footnote{\url{https://github.com/clef-men/zoo}}, following this guideline.
We review the works related to the verification of \OCaml programs in \cref{sec:related}. We describe our framework in \cref{sec:zoo}. We detail the important features, including the treatment of physical equality, in \cref{sec:features}. Finally we mention some side-contributions of this work, which are improvements to the \OCaml language and implementation to better support lock-free concurrent programs.

\begin{figure}[htb]
\begin{minted}{ocaml}
type state =
  | Open of Unix.file_descr
  | Closing of (unit -> unit)

type t =
  { mutable ops: int [@atomic];
    mutable state: state [@atomic];
  }

let make fd =
  { ops= 0; state= Open fd }

let closed =
  Closing (fun () -> ())
let close t =
  match t.state with
  | Closing _ ->
      false
  | Open fd as prev ->
      let close () = Unix.close fd in
      let next = Closing close in
      if Atomic.Loc.compare_and_set [%atomic.loc t.state] prev next then (
        if t.ops == 0
        && Atomic.Loc.compare_and_set [%atomic.loc t.state] next closed
        then
          close () ;
        true
      ) else (
        false
      )
\end{minted}
\caption{\mintinline{ocaml}{Rcfd.close} function from \Eio~\cite{eio}}
\label{fig:rcfd}
\end{figure}


\section{Related work}
\label{sec:related}

In general there are two approaches to practical program verification:

\subsection{Non-automated verification}

The verified program is translated, manually or in an automated way, into a representation living inside a proof assistant.
The user has to write specifications and prove them.

The representation may be primitive, like Gallina for \Rocq.
For pure programs, this is rather straightforward, \eg in \texttt{hs-to-coq}~\cite{DBLP:conf/cpp/Spector-Zabusky18}.
For imperative programs, this is more challenging.
One solution is to use a monad, \eg in \texttt{coq-of-ocaml}~\cite{coq-of-ocaml}, but it does not support concurrency.

The representation may be embedded, meaning the semantics of the language is formalized in the proof assistant.
This is the path taken by some recent works~\cite{DBLP:books/hal/Chargueraud23, DBLP:journals/pacmpl/GondelmanHPTB23, DBLP:conf/sosp/ChajedTKZ19,osiris} harnessing the power of separation logic.
In particular, \CFML~\cite{DBLP:books/hal/Chargueraud23} and \Osiris~\cite{osiris} target \OCaml.
However, \CFML does not support concurrency and is not based on \Iris.
\Osiris, still under development, is based on \Iris but does not support concurrency.

At the time of writing, \HeapLang is thus the most appropriate tool to verify concurrent \OCaml programs. We discussed limitations of \HeapLang in the introduction, and \ZooLang is our proposal to improve on this. Conversely, one notable limitation of \ZooLang today is its lack of support for \OCaml's relaxed memory model.

\subsection{Semi-automated verification}

In semi-automated verification approaches, the verified program is annotated by the user to guide the verification tool: preconditions, postconditions, invariants, \etc.
Given this input, the verification tool generates proof obligations that are mostly automatically discharged.
One may further distinguish two types of semi-automated systems: \emph{foundational} and \emph{non-foundational}.

In \emph{non-foundational} automated verification, the tool and the external solvers it may rely on are part of the trusted computing base.
It is the most common approach and has been widely applied in the literature~\cite{DBLP:journals/jfp/SwamyCFSBY13, DBLP:series/natosec/0001SS17, DBLP:conf/nfm/JacobsSPVPP11, DBLP:conf/icfem/DenisJM22, DBLP:conf/nfm/AstrauskasBFGMM22, DBLP:conf/esop/FilliatreP13, DBLP:journals/pacmpl/LattuadaHCBSZHPH23, DBLP:journals/pacmpl/PulteMSMSK23}, including to \OCaml by \Cameleer~\cite{DBLP:conf/cav/PereiraR20}, which uses the \Gospel specification language~\cite{DBLP:conf/fm/ChargueraudFLP19} and \WhyThree~\cite{DBLP:conf/esop/FilliatreP13}.

In \emph{foundational} automated verification, the proofs are checked by a proof assistant like \Rocq, meaning the automation does not have to be trusted.
To our knowledge, it has been applied to \C~\cite{DBLP:conf/pldi/SammlerLKMD021} and \Rust~\cite{DBLP:journals/pacmpl/GaherSJKD24}.

\Zoo is a non-automated verification framework---except for our use \Diaframe for local automation of separation logic reasoning. We would be interested in moving towards more automation in the future.

\subsection{Physical equality}

There is some literature in proof-assistant research on reflecting physical equality from the implementation language into the proof assistant, for optimization purposes: for example, exposing \OCaml's physical equality as a predicate in \Rocq lets us implement some memoization and sharing techniques in \Rocq libraries.
%
However, axiomatizing physical equality in the proof assistant is difficult, and can result in inconsistencies.

The earlier discussions of this question that we know come from Jourdan's thesis~\cite{DBLP:phd/hal/Jourdan16} (chapter 9), also presented more succintly in \cite{DBLP:journals/jar/BraibantJM14}.
%
This work introduces the Jourdan condition, that physical equality implies equality of values.
%
\cite{boulme:tel-03356701} extends the treatment of physical equality in \Rocq, integrating it in an ``extraction monad'' to control it more safely.
%
There is also a discussion of similar optimizations in \Lean in \cite{lean-pointer-optimizations}.

The correctness of the axiomatization of physical equality depends on the type of the values being compared: axiomatizations are typically polymorphic on any type \mintinline{coq}{A}, but their correctness depends on the specific \mintinline{coq}{A} being considered.
%
For example, it is easy to correctly characterize physical on natural numbers, and other non-dependent types arising in \Rocq verification projects.
%
One difficulty in \HeapLang and \ZooLang is that they are untyped languages, their representation of \mintinline{ocaml}{0} and \mintinline{ocaml}{false} has the same type.
%
But our remark that structural equality (in \OCaml) does not necessarily coincide with definitional equality (in \Rocq) also applies to other \Rocq types: our examples with an existential \mintinline{ocaml}{Any} constructor (see~\cref{sec:physical_equality}) can be reproduced with $\Sigma$-types.


\section{\Zoo in practice}
\label{sec:zoo}

\begin{figure}[htb]
\centering
\begin{tabular}{llcl}
    \Rocq term &
    $t$
  \\
    constructor &
    $C$
  \\
    projection &
    $\mathit{proj}$
  \\
    record field &
    $\mathit{fld}$
  \\
    identifier &
    $s, f$
    & $\in$ &
    $\mathrm{String}$
  \\
    integer &
    $n$
    & $\in$ &
    $\mathbb{Z}$
  \\
    boolean &
    $b$
    & $\in$ &
    $\mathbb{B}$
  \\
    binder &
    $x$
    & $\Coloneqq$ &
    $\texttt{<>} \mid s$
  \\
    unary operator &
    $\oplus$
    & $\Coloneqq$ &
    $\texttt{\raisebox{0.5ex}{\texttildelow}} \mid \texttt{-}$
  \\
    binary operator &
    $\otimes$
    & $\Coloneqq$ &
    $\texttt{+} \mid \texttt{-} \mid \texttt{*} \mid \texttt{`quot`} \mid \texttt{`rem`} \mid \texttt{`land`} \mid \texttt{`lor`} \mid \texttt{`lsl`} \mid \texttt{`lsr`}$
  \\
    && | &
    $\texttt{<=} \mid \texttt{<} \mid \texttt{>=} \mid \texttt{>} \mid \texttt{=} \mid \texttt{≠} \mid \texttt{==} \mid \texttt{!=}$
  \\
    && | &
    $\texttt{and} \mid \texttt{or}$
  \\
    expression &
    $e$
    & $\Coloneqq$ &
    $t \mid s \mid \texttt{\#} n \mid \texttt{\#} b$
  \\
    && | &
    $\texttt{fun:}\ x_1 \dots x_n\ \texttt{=>}\ e \mid \texttt{rec:}\ f\ x_1 \dots x_n\ \texttt{=>}\ e$
  \\
   && | &
   $\texttt{let:}\ x\ \texttt{:=}\ e_1\ \texttt{in}\ e_2 \mid e_1\ \texttt{;;}\ e_2$
  \\
    && | &
    $\texttt{let:}\ f\ x_1 \dots x_n\ \texttt{:=}\ e_1\ \texttt{in}\ e_2 \mid \texttt{letrec:}\ f\ x_1 \dots x_n\ \texttt{:=}\ e_1\ \texttt{in}\ e_2$
  \\
    && | &
    $\texttt{let:}\ \texttt{‘} C\ x_1 \dots x_n\ \texttt{:=}\ e_1\ \texttt{in}\ e_2 \mid \texttt{let:}\ x_1 \texttt{,} \dots \texttt{,} x_n\ \texttt{:=}\ e_1\ \texttt{in}\ e_2$
  \\
    && | &
    $\oplus e \mid e_1 \otimes e_2$
  \\
    && | &
    $\texttt{if:}\ e_0\ \texttt{then}\ e_1\ (\texttt{else}\ e_2)^?$
  \\
    && | &
    $\texttt{for:}\ x\ \texttt{:=}\ e_1\ \texttt{to}\ e_2\ \texttt{begin}\ e_3\ \texttt{end}$
  \\
    && | &
    $\texttt{§}C \mid \texttt{‘} C\ \texttt{(} e_1 \texttt{,} \dots \texttt{,} e_n \texttt{)} \mid \texttt{(} e_1 \texttt{,} \dots \texttt{,} e_n \texttt{)} \mid e \texttt{.<} \mathit{proj} \texttt{>}$
  \\
    && | &
    $\texttt{[]} \mid e_1\ \texttt{::}\ e_2$
  \\
    && | &
    $\texttt{‘} C\ \texttt{\{} e_1 \texttt{,} \dots \texttt{,} e_n \texttt{\}} \mid \texttt{\{} e_1 \texttt{,} \dots \texttt{,} e_n \texttt{\}} \mid e \texttt{.\{} \mathit{fld} \texttt{\}} \mid e_1\ \texttt{<-\{} \mathit{fld} \texttt{\}}\ e_2$
  \\
    && | &
    $\texttt{ref}\ e \mid \texttt{!} e \mid e_1\ \texttt{<-}\ e_2$
  \\
    && | &
    $\texttt{match:}\ e_0\ \texttt{with}\ \mathit{br}_1 \texttt{|} \dots \texttt{|}\ \mathit{br}_n\ (\texttt{|\_}\ (\texttt{as}\ s)^?\ \texttt{=>}\ e)^?\ \texttt{end}$
  \\
    && | &
    $e \texttt{.[} \mathit{fld} \texttt{]} \mid \texttt{Xchg}\ e_1\ e_2 \mid \texttt{CAS}\ e_1\ e_2\ e_3 \mid \texttt{FAA}\ e_1\ e_2$
  \\
    && | &
    $\texttt{Proph} \mid \texttt{Resolve}\ e_0\ e_1\ e_2$
  \\
    branch &
    $\mathit{br}$
    & $\Coloneqq$ &
    $C\ (x_1 \dots x_n)^?\ (\texttt{as}\ s)^?\ \texttt{=>}\ e$
  \\
    && | &
    $\texttt{[]}\ (\texttt{as}\ s)^?\ \texttt{=>}\ e \mid x_1\ \texttt{::}\ x_2\ (\texttt{as}\ s)^?\ \texttt{=>}\ e$
  \\
    toplevel value &
    $v$
    & $\Coloneqq$ &
    $t \mid \texttt{\#} n \mid \texttt{\#} b$
  \\
    && | &
    $\texttt{fun:}\ x_1 \dots x_n\ \texttt{=>}\ e \mid \texttt{rec:}\ f\ x_1 \dots x_n\ \texttt{=>}\ e$
  \\
    && | &
    $\texttt{§}C \mid \texttt{‘} C\ \texttt{(} v_1 \texttt{,} \dots \texttt{,} v_n \texttt{)} \mid \texttt{(} v_1 \texttt{,} \dots \texttt{,} v_n \texttt{)}$
  \\
    && | &
    $\texttt{[]} \mid v_1\ \texttt{::}\ v_2$
\end{tabular}
\caption{\ZooLang syntax (omitting mutually recursive toplevel functions)}
\label{fig:zoo}
\end{figure}

% In this section, we give an overview of our framework.
% We also provide a minimal example\footnote{\urlAnonymous{https://github.com/clef-men/zoo-demo}} demonstrating its use.

\subsection{Language}

The core of \Zoo is \ZooLang: a concurrent, imperative, untyped, functional programming language fully formalized in \Rocq.
Its semantics has been designed to match \OCaml's.

\ZooLang comes with a program logic based on \Iris: reasoning rules expressed in separation logic (including rules for the different constructs of the language) along with \Rocq tactics that integrate into the \Iris proof mode~\cite{DBLP:conf/popl/KrebbersTB17,DBLP:journals/pacmpl/KrebbersJ0TKTCD18}.
In addition, it supports \Diaframe~\cite{DBLP:conf/pldi/MulderKG22,DBLP:journals/pacmpl/MulderK23}, enabling proof automation.

The \ZooLang syntax is given in \cref{fig:zoo}\footnote{More precisely, it is the syntax of the surface language, including \Rocq notations.}, omitting mutually recursive toplevel functions that are treated specifically.
Expressions include standard constructs like booleans, integers, anonymous functions (that may be recursive), applications, \mintinline{ocaml}{let} bindings, sequence, unary and binary operators, conditionals, \mintinline{ocaml}{for} loops, tuples.
In any expression, one can refer to a \Rocq term representing a \ZooLang value (of type \mintinline{coq}{val}) using its \Rocq identifier.
\ZooLang is deeply embedded: variables (bound by functions and \mintinline{ocaml}{let}) are quoted as strings.

Data constructors (immutable memory blocks) are supported through two constructs : $\texttt{§}C$ represents a constant constructor (\eg $\texttt{§}\texttt{None}$), $\texttt{‘} C\ \texttt{(} e_1 \texttt{,} \dots \texttt{,} e_n \texttt{)}$ represents a non-constant constructor (\eg $\texttt{‘} \texttt{Some( } e \texttt{ )}$).
Unlike \OCaml, \ZooLang has projections of the form $e \texttt{.<} \mathit{proj} \texttt{>}$ (\eg $\texttt{(} x, y \texttt{).<1>}$), that can be used to obtain a specific component of a tuple or data constructor.
\ZooLang supports shallow pattern matching (patterns cannot be nested) on data constructors with an optional fallback case.

Mutable memory blocks are constructed using either the untagged record syntax $\texttt{\{} e_1 \texttt{,} \dots \texttt{,} e_n \texttt{\}}$ or the tagged record syntax $\texttt{‘} C\ \texttt{\{} e_1 \texttt{,} \dots \texttt{,} e_n \texttt{\}}$.
Reading a record field can be performed using $e \texttt{.\{} \mathit{fld} \texttt{\}}$ and writing to a record field using $e_1\ \texttt{<-\{} \mathit{fld} \texttt{\}}\ e_2$.
Pattern matching can also be used on mutable tagged blocks provided that cases do not bind anything---in other words, only the tag is examined, no memory access is performed.
References are also supported through the usual constructs : $\texttt{ref}\ e$ creates a reference, $\texttt{!} e$ reads a reference and $e_1\ \texttt{<-}\ e_2$ writes into a reference.
The syntax seemingly does not include constructs for arrays but they are supported through the \mintinline{ocaml}{Array} standard module (\eg \texttt{array\_make}).

Note that \ZooLang follows \OCaml in sometimes eschewing orthogonality to provide more compact memory representations: constructors are $n$-ary instead of taking a tuple as parameter, and the tagged record syntax is distinct from a constructor taking a mutable record as parameter. In each case the simplifying encoding would introduce an extra indirection in memory, which is absent from the \ZooLang semantics. Performance-conscious experts care about these representation choices, and we care about faithfully modeling their programs.

Parallelism is mainly supported through the \mintinline{ocaml}{Domain} standard module (\eg \texttt{domain\_spawn}), including domain-local storage.
Special constructs (\texttt{Xchg}, \texttt{CAS}, \texttt{FAA}; see \cref{sec:atomic}) are used to model atomic references.

The \texttt{Proph} and \texttt{Resolve} constructs model \emph{prophecy variables}~\cite{DBLP:journals/pacmpl/JungLPRTDJ20}, see \cref{sec:prophecy}.

\subsection{Translation from \OCaml to \ZooLang}

\begin{figure}[htb]
\begin{minted}{ocaml}
type 'a t =
  'a list Atomic.t

let create () =
  Atomic.make []

let rec push t v =
  let old = Atomic.get t in
  let new_ = v :: old in
  if not @@ Atomic.compare_and_set t old new_ then (
    Domain.cpu_relax () ;
    push t v
  )

let rec pop t =
  match Atomic.get t with
  | [] ->
      None
  | v :: new_ as old ->
      if Atomic.compare_and_set t old new_ then (
        Some v
      ) else (
        Domain.cpu_relax () ;
        pop t
      )
\end{minted}
\caption{Implementation of a concurrent stack}
\label{fig:stack}
\end{figure}

While \ZooLang lives in \Rocq, we want to verify \OCaml programs.
To connect them we provide the tool \texttt{ocaml2zoo} to translate \OCaml source files\footnote{Actually, \texttt{ocaml2zoo} processes binary annotation files (\texttt{.cmt} files).} into \Rocq files containing \ZooLang code.
This tool can process entire \texttt{dune} projects, and support several libraries provided together or as dependencies of the project.

The supported \OCaml fragment includes: tuples, variants, records and inline records, shallow \mintinline{ocaml}{match}, atomic record fields, unboxed types, toplevel mutually recursive functions.

Consider, for example, the \OCaml implementation of a concurrent stack~\cite{thomas1986systems} in \cref{fig:stack}.
The \mintinline{ocaml}{push} function is translated into:

\begin{minted}{coq}
Definition stack_push : val := rec: "push" "t" "v" =>
  let: "old" := !"t" in
  let: "new_" := "v" :: "old" in
  if: ~ CAS "t".[contents] "old" "new_" then (
    domain_cpu_relax () ;;
    "push" "t" "v" ).
\end{minted}

\subsection{Specifications and proofs}

Once the translation to \ZooLang is done, the user can write specifications and prove them in \Iris.
For instance, the specification of the \mintinline{ocaml}{stack_push} function could be:

\begin{minted}{coq}
Lemma stack_push_spec t ι v :
  <<< stack_inv t ι
    | ∀∀ vs, stack_model t vs >>>
    stack_push t v @ ↑ι
  <<< stack_model t (v :: vs)
    | RET (); True >>>.
Proof. ... Qed.
\end{minted}

Here, we use a \emph{logically atomic specification}~\cite{DBLP:conf/ecoop/PintoDG14}, which has been proven~\cite{DBLP:journals/pacmpl/BirkedalDGJST21} to be equivalent to \emph{linearizability}~\cite{DBLP:journals/toplas/HerlihyW90} in sequentially consistent memory models.

Similarly to \href{https://en.wikipedia.org/wiki/Hoare_logic}{Hoare triples},
the specification is formed of a precondition and a postcondition, represented in angular brackets.
But each is split in two parts, a \emph{public} or \emph{atomic} condition, and a \emph{private} condition.
Following standard \Iris notations, the private conditions are on the outside (first line of the precondition, last line of the postcondition) and the atomic conditions are inside.

For this particular operation, the private postcondition is trivial.
The private condition $\mathtt{stack\_inv}\ t$ is the stack invariant.
Intuitively, it asserts that $t$ is a valid concurrent stack.
More precisely, it enforces a set of logical constraints---a concurrent protocol---that $t$ must respect at all times.

The atomic pre- and post-conditions specify the linearization point of the operation: during the execution of \texttt{stack\_push}, the abstract state of the stack held by $\mathtt{stack\_model}$ is atomically updated from $\mathit{vs}$ to $v :: \mathit{vs}$: $v$ is atomically pushed at the top of the stack.


\section{\Zoo features}
\label{sec:features}

In this section, we review the salient features of \Zoo, which we found lacking when we attempted to use \HeapLang to verify real-world \OCaml programs.
We start with the most generic ones and then address those related to concurrency.

\subsection{Algebraic data types}

\Zoo is an untyped language but, to write interesting programs, it is convenient to work with abstractions like algebraic data types.
To simulate tuples, variants and records, we designed a machinery to define projections, constructors and record fields.

For example, one may define a list-like type with:

\begin{minted}{coq}
Notation "'Nil'"  := (in_type "t" 0) (in custom zoo_tag).
Notation "'Cons'" := (in_type "t" 1) (in custom zoo_tag).
\end{minted}

Users do not need to write this incantation directly, as they are generated by \texttt{ocaml2zoo} from the \OCaml type declarations.
Suffice it to say that it introduces the two tags in the \texttt{zoo\_tag} custom entry, on which the notations for data constructors rely.
The \mintinline{coq}{in_type} term is needed to distinguish the tags of distinct data types; crucially, it cannot be simplified away by \Rocq, as this could lead to confusion during the reduction of expressions.

Given this incantation, one may directly use the tags \texttt{Nil} and \texttt{Cons} in data constructors using the corresponding \ZooLang constructs:

\begin{minted}{coq}
Definition map : val :=
  rec: "map" "fn" "t" =>
    match: "t" with
    | Nil =>
        §Nil
    | Cons "x" "t" =>
        let: "y" := "fn" "x" in
        ‘Cons( "y", "map" "fn" "t" )
    end.
\end{minted}

Similarly, one may define a record-like type with two mutable fields \texttt{f1} and \texttt{f2}:

\begin{minted}{coq}
Notation "'f1'" := (in_type "t" 0) (in custom zoo_field).
Notation "'f2'" := (in_type "t" 1) (in custom zoo_field).

Definition swap : val :=
  fun: "t" =>
    let: "f1" := "t".{f1} in
    "t" <-{f1} "t".{f2} ;;
    "t" <-{f2} "f1".
\end{minted}

\subsection{Mutually recursive functions}

\Zoo supports non-recursive ($\texttt{fun:}\ x_1 \dots x_n\ \texttt{=>}\ e$) and recursive ($\texttt{rec:}\ f\ x_1 \dots x_n\ \texttt{=>}\ e$) functions but only \emph{toplevel} mutually recursive functions.
It is non-trivial to properly handle mutual recursion: when applying a mutually recursive function, a naive approach would replace calls to sibling functions by their respective bodies, but this typically makes the resulting expression unreadable.
To prevent it, the mutually recursive functions have to know one another to preserve their names during $\beta$-reduction.
We simulate this using some boilerplate that can be generated by \texttt{ocaml2zoo}.
For instance, one may define two mutually recursive functions \texttt{f} and \texttt{g} as follows:

\begin{minted}{coq}
Definition f_g := (
  recs: "f" "x" => "g" "x"
  and:  "g" "x" => "f" "x"
)%zoo_recs.

(* boilerplate *)
Definition f := ValRecs 0 f_g.
Definition g := ValRecs 1 f_g.
Instance : AsValRecs' f 0 f_g [f;g]. Proof. done. Qed.
Instance : AsValRecs' g 1 f_g [f;g]. Proof. done. Qed.
\end{minted}

\subsection{Standard library}

To save users from reinventing the wheel, we provide a standard library---more or less a subset of the \OCaml standard library.
Currently, it mainly includes standard data structures like: array (\mintinline{ocaml}{Array}), resizable array (\mintinline{ocaml}{Dynarray}), list (\mintinline{ocaml}{List}), stack (\mintinline{ocaml}{Stack}), queue (\mintinline{ocaml}{Queue}), double-ended queue, mutex (\mintinline{ocaml}{Mutex}), condition variable (\mintinline{ocaml}{Condition}).

Each of these standard modules contains \ZooLang functions and their verified specifications.
These specifications are modular: they can be used to verify more complex data structures.
As an evidence of this, lists~\citeAnonymous{DBLP:journals/pacmpl/AllainC0S24} and arrays~\citeAnonymous{allain:hal-04681703} have been successfully used in verification efforts based on \Zoo.

\subsection{Concurrent primitives}
\label{sec:atomic}

\Zoo supports concurrent primitives both on atomic references (from \mintinline{ocaml}{Atomic}) and atomic record fields (from \mintinline{ocaml}{Atomic.Loc}\footnote{The \mintinline{ocaml}{Atomic.Loc} module is part of the \href{https://github.com/ocaml/ocaml/pull/13404}{PR} that implements atomic record fields.}) according to the table below.
The \OCaml expressions listed in the left-hand column translate into the \Zoo expressions in the right-hand column.
Notice that an atomic location \mintinline[escapeinside=||]{ocaml}{[%atomic.loc |$e$|.|$f$|]} (of type \mintinline{ocaml}{_ Atomic.Loc.t}) translates directly into $e \texttt{.[} f \texttt{]}$.

\begin{center}
\begin{tabular}{ll}
    \OCaml &
    \Zoo
  \\ \hline
    \mintinline[escapeinside=||]{ocaml}{Atomic.get |$e$|} &
    $\texttt{!} e$
  \\
    \mintinline[escapeinside=||]{ocaml}{Atomic.set |$e_1$| |$e_2$|} &
    $e_1\ \texttt{<-}\ e_2$
  \\
    \mintinline[escapeinside=||]{ocaml}{Atomic.exchange |$e_1$| |$e_2$|} &
    $\texttt{Xchg}\ e_1 \texttt{.[contents]}\ e_2$
  \\
    \mintinline[escapeinside=||]{ocaml}{Atomic.compare_and_set |$e_1$| |$e_2$| |$e_3$|} &
    $\texttt{CAS}\ e_1 \texttt{.[contents]}\ e_2\ e_3$
  \\
    \mintinline[escapeinside=||]{ocaml}{Atomic.fetch_and_add |$e_1$| |$e_2$|} &
    $\texttt{FAA}\ e_1 \texttt{.[contents]}\ e_2$
  \\
    \mintinline[escapeinside=||]{ocaml}{Atomic.Loc.exchange [%atomic.loc |$e_1$|.|$f$|] |$e_2$|} &
    $\texttt{Xchg}\ e_1 \texttt{.[} f \texttt{]}\ e_2$
  \\
    \mintinline[escapeinside=||]{ocaml}{Atomic.Loc.compare_and_set [%atomic.loc |$e_1$|.|$f$|] |$e_2$| |$e_3$|} &
    $\texttt{CAS}\ e_1 \texttt{.[} f \texttt{]}\ e_2\ e_3$
  \\
    \mintinline[escapeinside=||]{ocaml}{Atomic.Loc.fetch_and_add [%atomic.loc |$e_1$|.|$f$|] |$e_2$|} &
    $\texttt{FAA}\ e_1 \texttt{.[} f \texttt{]}\ e_2$
\end{tabular}
\end{center}

One important aspect of this translation is that atomic accesses (\mintinline{ocaml}{Atomic.get} and \mintinline{ocaml}{Atomic.set}) correspond to plain loads and stores.
This is because we are working in a sequentially consistent memory model: there is no difference between atomic and non-atomic memory locations.

\subsection{Prophecy variables}
\label{sec:prophecy}

Lock-free algorithms exhibit complex behaviors.
To tackle them, \Iris provides powerful mechanisms such as \emph{prophecy variables}~\cite{DBLP:journals/pacmpl/JungLPRTDJ20}.
Essentially, prophecy variables can be used to predict the future of the program execution and reason about it.
They are key to handle \emph{future-dependent linearization points}: linearization points that may or may not occur at a given location in the code depending on a future observation.

\Zoo supports prophecy variables through the \texttt{Proph} and \texttt{Resolve} expressions---as in \HeapLang, the canonical \Iris language.
In \OCaml, these expressions correspond to \mintinline{ocaml}{Zoo.proph} and \mintinline{ocaml}{Zoo.resolve}, that are recognized by \texttt{ocaml2zoo}.


\section{Physical equality}
\label{sec:physical_equality}

The notion of \emph{physical equality} is ubiquitous in fine-grained concurrent algorithms.
It appears not only in the semantics of the \mintinline{ocaml}{==} operator, but also in the semantics of the \mintinline{ocaml}{Atomic.compare_and_set} primitive, which atomically sets an atomic reference to a desired value if its current content is physically equal to an expected value.
This primitive is commonly used to try committing an atomic operation in a retry loop, as in the \mintinline{ocaml}{push} and \mintinline{ocaml}{pop} functions of \cref{fig:stack}.

\subsection{Physical equality in \HeapLang}

In \HeapLang, this primitive is provided but restricted.
Indeed, its semantics is only defined if either the expected or the desired value fits in a single memory word in the \HeapLang value representation: literals (booleans, integers and pointers\footnote{\HeapLang allows arbitrary pointer arithmetic and therefore inner pointers. This is forbidden in both \OCaml and \ZooLang, as any reachable value has to be compatible with the garbage collector.}) and literal injections\footnote{\HeapLang has no primitive notion of constructor, only pairs and injections (left and right).}; otherwise, the program is stuck.
In practice, this restriction forces the programmer to introduce an indirection~\cite{iris/examples,DBLP:journals/pacmpl/JungLPRTDJ20,DBLP:conf/cpp/VindumB21} to physically compare complex values, \eg lists.
Furthermore, when the semantics is defined, values are compared using their \Rocq representations; physical equality boils down to \Rocq equality.

\subsection{Physical equality in \OCaml}

In \OCaml, physical equality is more tricky and often considered dangerous.
\emph{Structural equality}, which we describe in \cref{sec:structural_equality}, should be the preferred way of comparing values.
However, structural equality is typically much slower than physical equality, as it basically compiles to only one assembly instruction.
Also, the \mintinline{ocaml}{Atomic.compare_and_set} requires the comparison to be atomic, which is the case for physical equality but not structural equality.

In particular, the semantics of physical equality is \emph{non-deterministic}.
To see why, consider the case of \emph{immutable blocks} representing constructors and immutable records (as opposed to \emph{mutable blocks} representing mutable records), \eg \mintinline{ocaml}{Some 0}.
The physical comparison of two seemingly identical immutable blocks, according to the \Rocq representation (essentially a tag and a list of fields), may return \mintinline{ocaml}{false}.
Indeed, at runtime, a non-empty immutable block is represented by a pointer to a tagged memory block.
In this case, physical equality is just pointer comparison.
It is clear that two pointers being distinct does not imply the pointed memory blocks are.
In other words, we cannot determine the result of physical comparison just by looking at the abstract values.

The question is then: what guarantees do we get when physical equality returns \mintinline{ocaml}{true} and when it returns \mintinline{ocaml}{false}?
The \OCaml manual documents a partial specification for physical equality, which is precise for basic types such as references, but does not clearly extend to structured values containing a mix of immutable and mutable constructors.
The only guarantee that it provides for all values is: if two values are physically equal, they are also structurally equal.
This means we don't learn anything when two values are physically distinct.

In the following, we will explore both cases, looking at the optimizations that the compiler or the runtime system may perform.
We will show that the aforementioned guarantee is arguably not sufficient to verify interesting concurrent programs and attempt to establish stronger guarantees.

\subsection{When physical equality returns \mintinline{ocaml}{true}}

Let us go back to the concurrent stack of \cref{fig:stack} and more specifically the \mintinline{ocaml}{push} function.
To prove the atomic specification given in \cref{sec:zoo}, we rely on the fact that, if \mintinline{ocaml}{Atomic.compare_and_set} returns \mintinline{ocaml}{true}, we actually observe the same list of values in the sense of \Rocq equality.
However, assuming only structural equality as per \OCaml's specification of physical equality, this cannot be proven.
To see why, consider, \eg, a stack of references (\mintinline{ocaml}{'a ref}).
As structural equality is indeed \emph{structural}, it traverses the references without comparing their \emph{physical identities}.
In other words, we cannot conclude the references are \emph{exactly} the same.
Hence, we cannot prove the specification.

This conclusion might seem surprising and counterintuitive.
Indeed, we know that physical equality essentially boils down to a comparison instruction, so we should be able to say more.
Departing from \OCaml's imprecise specification, let us attempt to establish stronger guarantees.
We assume the following classification of values: booleans, integers, mutable blocks (pointers), immutable blocks, functions.

The easy cases are mutable blocks and functions.
Each of these two classes is disjoint from the others.
We can reasonably assume that, when physical equality returns \mintinline{ocaml}{true} and one of the compared values belongs to either of these classes, the two values are actually the same in \Rocq.
As far as we are aware, there is no optimization that could break this.

Booleans, integers and empty immutable blocks are represented by immediate integers through an encoding.
This encoding induces conflicts: two seemingly distinct values in \Rocq may have the same encoding.
For example, the following tests all return \mintinline{ocaml}{true} (\mintinline{ocaml}{Obj.repr} is an unsafe primitive revealing the memory representation of a value):
\begin{minted}{ocaml}
let test1 = Obj.repr false == Obj.repr 0 (* true *)
let test2 = Obj.repr None  == Obj.repr 0 (* true *)
let test3 = Obj.repr []    == Obj.repr 0 (* true *)
\end{minted}

The semantics of unrestricted physical equality has to reflect these conflicts.
In our experience, restricting compared values similarly to typing is quite burdensome; the specification of polymorphic data structures using physical equality has to be systematically restricted.
In summary, when physical equality on immediate values returns \mintinline{ocaml}{true}, it is guaranteed that they have the same encoding.

Finally, let us consider the case of non-empty immutable blocks.
At runtime, they are represented by pointers to tagged memory blocks.
At first approximation, it is tempting to say that physically equal immutable blocks really are the same in \Rocq.
Alas, this is not true.
To explain why, we have to recall that the \OCaml compiler and the runtime system (\eg, through hash-consing) may perform \emph{sharing}: immutable blocks containing physically equal fields may be shared.
For example, the following tests may return \mintinline{ocaml}{true}:
\begin{minted}{ocaml}
let test1 = Some 0 == Some 0 (* true *)
let test2 = [0;1]  == [0;1]  (* true *)
\end{minted}

On its own, sharing is not a problem.
However, coupled with representation conflicts, it can be surprising.
Indeed, consider the \mintinline{ocaml}{any} type defined as:
\begin{minted}{ocaml}
type any = Any : 'a -> any
\end{minted}

The following tests may return \mintinline{ocaml}{true}:
\begin{minted}{ocaml}
let test1 = Any false == Any 0 (* true *)
let test2 = Any None  == Any 0 (* true *)
let test3 = Any []    == Any 0 (* true *)
\end{minted}

Now, going back to the \mintinline{ocaml}{push} function of \cref{fig:stack}, we have a problem.
Given a stack of \mintinline{ocaml}{any}, it is possible for the \mintinline{ocaml}{Atomic.compare_and_set} to observe a current list (\eg, \mintinline{ocaml}{[Any 0]}) physically equal to the expected list (\eg, \mintinline{ocaml}{[Any false]}) while these are actually distinct in \Rocq.
In short, the expected specification of \cref{sec:zoo} is incorrect.
To fix it, we would need to reason \emph{modulo physically equality}, which is non-standard and quite burdensome.

We believe this really is a shortcoming, at least from the verification perspective.
Therefore, we propose to extend \OCaml with \emph{generative immutable blocks}\footnote{\urlAnonymous{https://github.com/clef-men/ocaml/tree/generative_constructors}}.
These generative blocks are just like regular immutable blocks, except they cannot be shared.
Hence, if physical equality on two generative blocks returns \mintinline{ocaml}{true}, these blocks are necessarily equal in \Rocq.
At user level, this notion is materialized by \emph{generative constructors}.
For instance, to verify the expected \mintinline{ocaml}{push} specification, we can use a generative version of lists:
\begin{minted}{ocaml}
type 'a list =
  | Nil
  | Cons of 'a * 'a list [@generative]
\end{minted}

\subsection{When physical equality returns \mintinline{ocaml}{false}}

\begin{figure}[htb]
\begin{minted}{ocaml}
type state =
  | Open of Unix.file_descr
  | Closing of (unit -> unit)

type t =
  { mutable ops: int [@atomic];
    mutable state: state [@atomic];
  }

let make fd =
  { ops= 0; state= Open fd }

let closed =
  Closing (fun () -> ())
let close t =
  match t.state with
  | Closing _ ->
      false
  | Open fd as prev ->
      let close () = Unix.close fd in
      let next = Closing close in
      if Atomic.Loc.compare_and_set [%atomic.loc t.state] prev next then (
        if t.ops == 0
        && Atomic.Loc.compare_and_set [%atomic.loc t.state] next closed
        then
          close () ;
        true
      ) else (
        false
      )
\end{minted}
\caption{\mintinline{ocaml}{Rcfd.close} function from \Eio~\cite{eio}}
\label{fig:rcfd}
\end{figure}

The informal \OCaml specification does not give any guarantee when physical equality returns \mintinline{ocaml}{false}.
In most cases, including try loops, this is fine.
However, in some specific cases, more information is needed.

Consider the \mintinline{ocaml}{Rcfd} module from the \Eio~\cite{eio} library, an excerpt of which is given in \cref{fig:rcfd}\footnote{We make use of \emph{atomic record fields} as introduced in \cref{sec:atomic-record-fields}.}.
Thomas Leonard, its author, suggested that we verify this real-life example because of its intricate logical state.
However, we found out that it is also relevant regarding the semantics of physical equality.
Essentially, it consists in wrapping a file descriptor in a thread-safe way using reference-counting.
At creation in the \mintinline{ocaml}{make} function, the wrapper starts in the \mintinline{ocaml}{Open} state.
At some point, it can switch to the \mintinline{ocaml}{Closing} state in the \mintinline{ocaml}{close} function and can never go back to the \mintinline{ocaml}{Open} state.
Crucially, the \mintinline{ocaml}{Open} state does not change throughout the lifetime of the data structure.

The interest of \mintinline{ocaml}{Rcfd} lies in the \mintinline{ocaml}{close} function.
First, the function reads the state.
If this state is \mintinline{ocaml}{Closing}, it returns \mintinline{ocaml}{false}; the wrapper has been closed.
If this state is \mintinline{ocaml}{Open}, it tries to switch to the \mintinline{ocaml}{Closing} state using \mintinline{ocaml}{Atomic.Loc.compare_and_set}; if this attempt fails, it also returns \mintinline{ocaml}{false}.
In this particular case, we would like to prove that the wrapper has been closed, or equivalently that \mintinline{ocaml}{Atomic.Loc.compare_and_set} cannot have observed \mintinline{ocaml}{Open}.
Intuitively, this is true because there is only one \mintinline{ocaml}{Open}.

Obviously, we need some kind of guarantee related to the \emph{physical identity} of \mintinline{ocaml}{Open} when \mintinline{ocaml}{Atomic.Loc.compare_and_set} returns \mintinline{ocaml}{false}.
If \mintinline{ocaml}{Open} were a mutable block, we could argue that this block cannot be physically distinct from itself; no optimization we know of would allow that.
Unfortunately, it is an immutable block, and immutable blocks are subject to more optimizations.
In fact, something surprising but allowed\footnote{This has been confirmed by \OCaml experts developing the \Flambda backend.} by \OCaml can happen: \emph{unsharing}, the dual of sharing.
Indeed, any immutable block can be unshared, that is reallocated.
For example, the following test may theoretically return \mintinline{ocaml}{false}:
\begin{minted}{ocaml}
let x = Some 0
let test = x == x (* false *)
\end{minted}

Going back to \mintinline{ocaml}{Rcfd}, we have a problem: in the second branch, the \mintinline{ocaml}{Open} block corresponding to \mintinline{ocaml}{prev} could be unshared, which would make \mintinline{ocaml}{Atomic.Loc.compare_and_set} fail.
Hence, we cannot prove the expected specification.

To remedy this unfortunate situation, we propose to reuse the notions of generative immutable blocks, that we introduced to prevent sharing, to also forbid unsharing.
More precisely, each generative block is annotated with a \emph{logical identifier}\footnote{Actually, for practical reasons, we distinguish identified and unidentified generative blocks.} representing its physical identity, much like a pointer for a mutable block.
If physical equality on two generative blocks returns \mintinline{ocaml}{false}, the two identifiers are necessarily distinct.
Given this semantics, we can verify the \mintinline{ocaml}{close} function.
Indeed, if \mintinline{ocaml}{Atomic.Loc.compare_and_set} fails, we now know that the identifiers of the two blocks, if any, are distinct.
As there is only one \mintinline{ocaml}{Open} block whose identifier does not change, it cannot be the case that the current state is \mintinline{ocaml}{Open}, hence it is \mintinline{ocaml}{Closing} and we can conclude.
In practice, it suffices to patch the \mintinline{ocaml}{state} type:
\begin{minted}{ocaml}
type state =
  | Open of Unix.file_descr [@generative]
  | Closing of (unit -> unit)
\end{minted}

%\section{Physical equality -- alternative proposal}
%
%The notion of \emph{physical equality} is ubiquitous in fine-grained concurrent algorithms.
%It appears not only in the semantics of the \mintinline{ocaml}{==} operator, but also in the semantics of the \mintinline{ocaml}{Atomic.compare_and_set} primitive, which atomically sets an atomic reference to a desired value if its current content is physically equal to an expected value.
%This primitive is commonly used to try committing an atomic operation in a retry loop, as in the \mintinline{ocaml}{push} and \mintinline{ocaml}{pop} functions of \cref{fig:stack}.
%
%At the same time this notion is difficult to specify correctly, and this can result in dangerous gaps between the programming language used to write code and the semantics used for its verification.
%
%\ZooLang has a grammar of values, and most operations are specified by defining how they compute with \ZooLang values. Its definition may look as follows in Rocq (simplified slightly):
%\begin{minted}{coq}
%Inductive literal :=
%  | Bool (b : bool)
%  | Int (n : nat)
%  | Loc (l : location)
%  | Proph (pid : prophet_id)
%  | Poison.
%
%Inductive val :=
%  | Lit (lit : literal)
%  | Recs (i : nat) (recs : list (binder * binder * expr))
%  | Block (tag : nat) (vs : list val).
%\end{minted}
%
%For example, the value $'\texttt{Cons} (42, \texttt{§}\texttt{Nil})$ is represented in Rocq as \mintinline{coq}{Block 1 [Lit (Int 42), Block 0 []]}. Notice that immutable blocks are represented in Rocq using the \mintinline{coq}{Block} constructor directly, and \emph{not} as a location (\mintinline{coq}{Loc}) allocated on the heap. We use locations only for \emph{mutable} records. We would say that our representation of \ZooLang values is \emph{high-level}, as close to the surface syntax as reasonably possible. This distinction is important to make verification pleasant in practice, by reducing the number of locations and heap indirections that the programmer needs to work with during verification. A \ZooLang tuple is directly a tuple, etc., and this design decision of using high-level values is important to the verification experience.
%\Xclement{Not only that, but assuming full ownership of arguments of immutable blocks would be incorrect.}
%
%It is tempting to specify, as \HeapLang does, that physical equality decides equality between high-level values. This specification makes sense for immediate values (integers, booleans), and for mutable records which are compared by location. But it is incorrect on immutable blocks, and \HeapLang essentially does not specify its behavior on those values. Yet programmers use physical equality on immutable blocks in practice, as in our example of a Treiber stack of \cref{fig:stack}.
%
%Defining physical equality as equality of high-level values is problematic in two opposite ways:
%\begin{enumerate}
%\item Some distinct high-level values are physically equal in OCaml, for example \mintinline{ocaml}{0} and \mintinline{ocaml}{false}. Their type differ, but it possible to store them in an existential type where they can be compared for physical equality:
%\begin{minted}{ocaml}
%type any = Any : 'a -> any
%let test1 = Any false == Any 0 (* true *)
%\end{minted}
%This shows that even on immediate values, specifying physical equality as equality of high-level values is convenient but incorrect in practice.
%
%\item A deeper problem is that some \emph{definitionally equal} high-level values may be physically distinct.
%  For example, if \mintinline{ocaml}{x} is defined as the integer \mintinline{ocaml}{42}, then \mintinline{ocaml}{(x :: []) == (42 :: [])} may or may not  hold, depending on the OCaml implementation being used.
%  But one can prove that both arguments are definitionally equal in Rocq, so physical equality cannot be modeled by a Rocq function of type \mintinline{coq}{val -> val -> bool}.
%\end{enumerate}


\section{Structural equality}
\label{sec:structural_equality}

Structural equality is also supported.
More precisely, it is not part of the semantics of the language but axiomatized on top of it\footnote{We could also have implemented it in \ZooLang, but that would require more low-level primitives.}.
The reason is that it is in fact difficult to specify for arbitrary values.
In general, we have to compare graphs---which implies structural comparison may diverge.

Accordingly, the specification of $v_1\ \texttt{=}\ v_2$ requires the (partial) ownership of a \emph{memory footprint} corresponding to the union of the two compared graphs, giving the permission to traverse them safely.
If it terminates, the comparison decides whether the two graphs are isomorphic (modulo representation conflicts, as described in \cref{sec:physical_equality}).
In \Iris, this gives:

\begin{minted}{coq}
Axiom structeq_spec : ∀ v1 v2 footprint,
  val_traversable footprint v1 →
  val_traversable footprint v2 →
  {{{ structeq_footprint footprint }}}
    v1 = v2
  {{{ b, RET #b;
      structeq_footprint footprint ∗
      ⌜(if b then val_structeq else val_structneq) footprint v1 v2⌝ }}}.
\end{minted}

Obviously, this general specification is not very convenient to work with.
Fortunately, for abstract values (without any mutable part), we can prove a much simpler variant saying that structural equality boils down to physical equality:

\begin{minted}{coq}
Lemma structeq_spec_abstract v1 v2 :
  val_abstract v1 →
  val_abstract v2 →
  {{{ True }}}
    v1 = v2
  {{{ b, RET #b; ⌜(if b then val_physeq else val_physneq) v1 v2⌝ }}}
Proof. ... Qed.
\end{minted}


\section{\OCaml extensions for fine-grained concurrent programming}
\label{sec:ocaml}

Over the course of this work, we studied efficient fine-grained concurrent \OCaml programs written by experts.
This revealed various limitations of \OCaml in these domains, that those experts would work around using unsafe casts, often at the cost of both readability and memory-safety; and also some mismatches between their mental model of the semantics of \OCaml and the mental model used by the \OCaml compiler authors.
We worked on improving \OCaml itself to reduce these work-arounds or semantic mismatches.

\subsection{Atomic record fields}
\label{sec:atomic-record-fields}

\subsubsection{Before}

\OCaml~5 offers a type \mintinline{ocaml}{'a Atomic.t} of atomic references exposing sequentially-consistent atomic operations.
Data races on non-atomic mutable locations has a much weaker semantics and is generally considered a programming error.
For example, the Michael-Scott concurrent queue~\cite{DBLP:conf/podc/MichaelS96} relies on a linked list structure that could be defined as follows:

\begin{minted}{ocaml}
type 'a node = Nil | Cons of { value : 'a; next : 'a node Atomic.t }
\end{minted}

Performance-minded concurrency experts dislike this representation, because \mintinline{ocaml}{'a Atomic.t} introduces an indirection in memory: it is represented as a pointer to a block containing the value of type \mintinline{ocaml}{'a}.
Instead, they use something like the following:

\begin{minted}{ocaml}
type 'a node = Nil | Cons of { mutable next: 'a node; value: 'a }
let as_atomic : 'a node -> 'a node Atomic.t option = function
  | Nil -> None
  | (Next _) as record -> Some (Obj.magic record : 'a node Atomic.t)
\end{minted}

Notice that the \mintinline{ocaml}{next} field of the \mintinline{ocaml}{Cons} constructor has been moved first in the type declaration.
Because the \OCaml compiler respects field-declaration order in data layout, a value \mintinline{ocaml}!Cons { next; value }! has a similar low-level representation to a reference (atomic or not) pointing at \mintinline{ocaml}{next}, with an extra argument.
The code uses \mintinline{ocaml}{Obj.magic} to unsafely cast this value to an atomic reference, which appears to work as intended.

\mintinline{ocaml}{Obj.magic} is a shunned unsafe cast (the \OCaml equivalent of \mintinline{ocaml}{unsafe} or \mintinline{ocaml}{unsafePerformIO}).
It is very difficult to be confident about its usage given that it may typically violate assumptions made by the \OCaml compiler and optimizer.
In the example above, casting a two-fields record into a one-argument atomic reference may or may not be sound---but it gives measurable performance improvements on concurrent queue benchmarks. (TODO: benchmark to quantify the improvement.)

It is possible to statically forbid passing \mintinline{ocaml}{Nil} to \mintinline{ocaml}{as_atomic} to avoid error handling, by turning \mintinline{ocaml}{'a node} into a GADT indexed over it a type-level representation of its head constructor.
Examples of this pattern can be found in the \Kcas library by Vesa Karvonen.
It is difficult to write correctly and use, in particular as unsafe casts can sometimes hide type-errors in the intended static discipline.

Note that this unsafe approach only works for the first field of a record, so it is not applicable to records that hold several atomic fields, such as the toplevel record storing atomic \mintinline{ocaml}{front} and \mintinline{ocaml}{back} pointers for the concurrent queue.

\subsubsection{Atomic fields proposal}

We proposed a design for atomic record fields as an \OCaml language change proposal: RFC~\#39\footnote{\urlAnonymous{https://github.com/ocaml/RFCs/pull/39}}.
Declaring a record field atomic simply requires an \mintinline{ocaml}{[@atomic]} attribute---and could eventually become a proper keyword of the language.

\begin{minted}{ocaml}
(* re-implementation of atomic references *)
type 'a atomic_ref = { mutable contents : 'a [@atomic]; }

(* concurrent linked list *)
type 'a node = Nil | Cons of { value: 'a; mutable next : 'a node [@atomic]; }

(* bounded SPSC circular buffer *)
type 'a bag =
  { data : 'a Atomic.t array;
    mutable front: int [@atomic];
    mutable back: int [@atomic]; }
\end{minted}

The design difficulty is to express atomic operations on atomic record fields.
For example, if \mintinline{ocaml}{buf} has type \mintinline{ocaml}{'a bag} above, then one naturally expects the existing notation \mintinline{ocaml}{buf.front} to perform an atomic read and \mintinline{ocaml}{buf.front <- n} to perform an atomic write.
But how would one express exchange, compare-and-set and fetch-and-add?
We would like to avoid adding a new primitive language construct for each atomic operation.

Our proposed implementation\footnote{\urlAnonymous{https://github.com/ocaml/ocaml/pull/13404}}
introduces a built-in type \mintinline{ocaml}{'a Atomic.Loc.t}
for an atomic location that holds an element of type
\mintinline{ocaml}{'a}, with a syntax extension
\mintinline{ocaml}{[%atomic.loc <expr>.<field>]}
to construct such locations. Atomic primitives operate on values of type \mintinline{ocaml}{'a Atomic.Loc.t},
and they are exposed as functions of the module \mintinline{ocaml}{Atomic.Loc}.

For example, the standard library exposes
\begin{minted}{ocaml}
val Atomic.Loc.fetch_and_add : int Atomic.Loc.t -> int -> int
\end{minted}
and users can write:
\begin{minted}{ocaml}
let preincrement_front (buf : 'a bag) : int =
  Atomic.Loc.fetch_and_add [%atomic.loc buf.front] 1
\end{minted}
where \mintinline{ocaml}{[%atomic.loc buf.front]} has type \mintinline{ocaml}{int Atomic.Loc.t}.
Internally, a value of type \mintinline{ocaml}{'a Atomic.Loc.t} can be represented as a pair of a record and an integer offset for the desired field, and the \mintinline{ocaml}{atomic.loc} construction builds this pair in a well-typed manner.
When a primitive of the \mintinline{ocaml}{Atomic.Loc} module is applied to an \mintinline{ocaml}{atomic.loc} expression, the compiler can optimize away the construction of the pair---but it would happen if there was an abstraction barrier between the construction and its use.

Note: the type \mintinline{ocaml}{'a Atomic.t} of atomic references exposes a function
\begin{minted}{ocaml}
val Atomic.make_contended : 'a -> 'a Atomic.t
\end{minted}
that ensures that the returned atomic value is allocated with enough alignment and padding to sit alone on its cache line, to avoid performance issues caused by false sharing.
Currently there is no such support for padding of atomic record fields (we are planning to
work on this if the support for atomic fields gets merged in standard \OCaml), so the less-compact atomic references remain preferable in certain scenarios.

\subsection{Atomic arrays}

On top of our atomic record fields, we have implemented support for atomic arrays, another facility commonly requested by authors of efficient concurrent programs.
Our previous example of a concurrent bag of type \mintinline{ocaml}{'a bag} used a backing array of type \mintinline{ocaml}{'a Atomic.t array}, which contains more indirections than may be desirable, as each array element is a pointer to a block containing the value of type \mintinline{ocaml}{'a}, instead of storing the value of type \mintinline{ocaml}{'a} directly in the array.

Our implementation of atomic arrays\footnote{\urlAnonymous{https://github.com/clef-men/ocaml/tree/atomic\_array}} builds on top of the type \mintinline{ocaml}{'a Atomic.Loc.t} we described in the previous section, and it relies on two new low-level primitives provided by the compiler:

\begin{minted}{ocaml}
val Atomic_array.index : 'a array -> int -> 'a Atomic.Loc.t
val Atomic_array.unsafe_index : 'a array -> int -> 'a Atomic.Loc.t
\end{minted}

The function \mintinline{ocaml}{index} takes an array and an integer index within the array, and returns an atomic location into the corresponding element after performing a bound check.
\mintinline{ocaml}{unsafe_index} omits the boundcheck---additional performance at the cost of memory-safety---and allows to express the atomic counterpart of the unsafe operations \mintinline{ocaml}{Array.unsafe_get} and \mintinline{ocaml}{Array.unsafe_set}.
The atomic primitives of the module \mintinline{ocaml}{Atomic.Loc} can then be used on these indices; our implementation implements a library module on top of these primitives to provide a higher-level layer to the user, with direct array operations such as:

\begin{minted}{ocaml}
val Atomic_array.exchange : 'a Atomic_array.t -> int -> 'a -> 'a
val Atomic_array.unsafe_exchange : 'a Atomic_array.t -> int -> 'a -> 'a
\end{minted}


\section{Conclusion and future work}

The development of \Zoo is still ongoing.
While it is not yet available on \texttt{opam}, it can be installed and used in other \Rocq projects.
We provide a \href{https://github.com/clef-men/zoo_demo}{minimal example} demonstrating its use.

\Zoo supports a limited fragment of \OCaml that is sufficient for most of our needs.
Its main weakness so far is its memory model, which is sequentially consistent as opposed to the relaxed \OCaml~5 memory model.
It also lacks exceptions and algebraic effects, that we plan to introduce in the future.

Another interesting direction would be to combine \Zoo with semi-automated techniques.
Similarly to \WhyThree, the simple parts of the verification effort would be done in a semi-automated way, while the most difficult parts would be conducted in \Rocq.



% ------------------------------------------------

\hbadness=10000
\bibliography{main}

\end{document}
