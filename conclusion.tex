\section{Conclusion and future work}

The development of \Zoo is still ongoing.
While it is not yet available on \texttt{opam}, it can be installed and used in other \Rocq projects.
We provide a \href{https://github.com/clef-men/zoo_demo}{minimal example} demonstrating its use.

\Zoo supports a limited fragment of \OCaml that is sufficient for most of our needs.
Its main weakness so far is its memory model, which is sequentially consistent as opposed to the relaxed \OCaml~5 memory model.
It also lacks exceptions and algebraic effects, that we plan to introduce in the future.

Another interesting direction would be to combine \Zoo with semi-automated techniques.
Similarly to \WhyThree, the simple parts of the verification effort would be done in a semi-automated way, while the most difficult parts would be conducted in \Rocq.

