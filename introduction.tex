\section{Introduction}
\label{sec:introduction}

Designing concurrent algorithms, in particular fine-grained concurrent algorithms, is a notoriously difficult task.
Similarly, the formal verification of such algorithms is also difficult.
It typically involves finding and reasoning about non-trivial linearization points~\cite{DBLP:journals/csur/DongolD15,DBLP:journals/pacmpl/JungLPRTDJ20,DBLP:conf/cpp/VindumB21,DBLP:conf/cpp/VindumFB22,DBLP:conf/osdi/Chang0STKZ23}.

In recent years, concurrent separation logic~\cite{DBLP:journals/siglog/BrookesO16} has enabled significant progress in this area.
In particular, the development of \Iris~\cite{DBLP:journals/jfp/JungKJBBD18}, a state-of-the-art mechanized \emph{higher-order} concurrent separation logic with \emph{user-defined ghost state}, has nourished a rich and successful line of works~\cite{DBLP:journals/pacmpl/JungLPRTDJ20,DBLP:conf/cpp/VindumB21,DBLP:conf/cpp/VindumFB22,DBLP:conf/osdi/Chang0STKZ23,DBLP:conf/cpp/CarbonneauxZKON22,DBLP:journals/pacmpl/JungLCKPK23,DBLP:journals/pacmpl/SomersK24,DBLP:journals/pacmpl/MevelJP20,DBLP:journals/pacmpl/MevelJ21,DBLP:conf/pldi/DangJCNMKD22,DBLP:journals/pacmpl/ParkKMJLKK24,DBLP:conf/pldi/MulderKG22,DBLP:journals/pacmpl/MulderK23}, dealing with external~\cite{DBLP:conf/cpp/VindumFB22} and future-dependent~\cite{DBLP:journals/pacmpl/JungLPRTDJ20,DBLP:conf/cpp/VindumB21,DBLP:conf/osdi/Chang0STKZ23} linearization points, relaxed memory~\cite{DBLP:journals/pacmpl/MevelJP20,DBLP:journals/pacmpl/MevelJ21,DBLP:conf/pldi/DangJCNMKD22,DBLP:journals/pacmpl/ParkKMJLKK24} and automation~\cite{DBLP:conf/pldi/MulderKG22,DBLP:journals/pacmpl/MulderK23}.

Most of these works~\cite{DBLP:journals/pacmpl/JungLPRTDJ20,DBLP:conf/cpp/VindumB21,DBLP:conf/cpp/VindumFB22,DBLP:conf/cpp/CarbonneauxZKON22,DBLP:journals/pacmpl/JungLCKPK23,DBLP:journals/pacmpl/SomersK24,DBLP:conf/pldi/MulderKG22,DBLP:journals/pacmpl/MulderK23} and many others~\cite{DBLP:journals/pacmpl/VilhenaPJ20,DBLP:journals/pacmpl/PottierGJM24,DBLP:journals/pacmpl/TimanyGB24,DBLP:journals/pacmpl/LorenzenLSL24} rely on \HeapLang~\cite{iris}, the exemplar \Iris language.
\HeapLang is a concurrent, imperative, untyped, call-by-value functional language.
To the best of our knowledge, it is currently the closest language to \OCaml~5 in the \Iris ecosystem---we review the existing frameworks in \cref{sec:related}.
It has been extended to handle weak memory~\cite{DBLP:journals/pacmpl/MevelJP20} and algebraic effects~\cite{DBLP:journals/pacmpl/VilhenaP21}.

Although \HeapLang is theoretically expressive enough to represent \OCaml programs, our experiments showed that it is fairly impractical when it comes to verifying large \OCaml libraries.
Indeed, it lacks basic abstractions such as algebraic data types (tuples, mutable and immutable records, variants) and mutually recursive functions.
Verifying \OCaml programs in \HeapLang requires difficult translation choices and introduces various encodings, to the point that the relation between the source and verified programs can become difficult to maintain and reason about.
It also has very few standard data structures that can be directly reused.
This view, we believe, is shared by many people in the \Iris community.
Our first motivation in this work is therefore to fill this gap by providing a more practical \OCaml-like verification language: \ZooLang.
This language consists in a subset of \OCaml~5 extended with atomic record fields and equipped with a formal semantics and a program logic based on \Iris.
We were influenced by the \Perennial~\cite{DBLP:conf/sosp/ChajedTKZ19,DBLP:conf/osdi/ChajedTT0KZ21,DBLP:conf/osdi/ChajedTTKZ22,DBLP:conf/osdi/Chang0STKZ23} framework, which achieved similar goals for the \Go language with a focus on crash-safety.
As in \Perennial, we also provide a translator from \OCaml to \ZooLang: \texttt{ocaml2zoo}.
We call the resulting framework \Zoo.

Another, maybe less obvious, shortcoming of \HeapLang is the soundness of its semantics with respect to \OCaml, in other words how faithful it is to the original language.
One ubiquitous---particularly in lock-free algorithms relying on low-level atomic primitives---and subtle point is \emph{physical equality}.
In \cref{sec:physical_equality}, we show that (1) \HeapLang's semantics for physical equality is not compatible with \OCaml and (2) \OCaml's informal semantics is actually too imprecise to verify basic concurrent algorithms.
To remedy this, we propose a new formal semantics for physical equality and structural equality.
We hope this work will influence the way these notions are specified in \OCaml.

In summary, we claim the following contributions:
\begin{enumerate}
  \item
    We present \ZooLang, a convenient subset of \OCaml~5 formalized in \Rocq (\cref{sec:zoo,sec:features}).
    \ZooLang comes with a program logic based on \Iris and supports proof automation through \Diaframe~\cite{DBLP:conf/pldi/MulderKG22,DBLP:journals/pacmpl/MulderK23}.
  \item
    We provide a translator from \OCaml to \ZooLang: \texttt{ocaml2zoo} (\cref{sec:zoo}), built for practical
    applications -- it supports full projects using the \texttt{dune} build system.
  \item
    We formalize physical equality (\cref{sec:physical_equality}) and structural equality (\cref{sec:structural_equality}) in a faithful way.
    The careful analysis of these notions suggests a new \OCaml feature: \emph{generative constructors}.
  \item
    We extend \OCaml with \emph{atomic record fields} and \emph{atomic arrays} to ease the development of fine-grained concurrent algorithms (\cref{sec:ocaml}).
  \item
    We verify realistic use cases (\cref{sec:physical_equality}) involving physical equality: (1) Treiber stack~\cite{thomas1986systems}, (2) a thread-safe wrapper around a file descriptor using reference-counting from the \Eio~\cite{eio} library.
\end{enumerate}
