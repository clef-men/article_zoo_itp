\section{Structural equality}
\label{sec:structural_equality}

%Structural equality is also supported.
%More precisely, it is not part of the semantics of the language but axiomatized on top of it\footnote{We could also have implemented it in \Zoo, but that would require more low-level primitives.}.
%The reason is that it is in fact difficult to specify for arbitrary values.
%Indeed, we have to handle not only abstract tree-like values (booleans, integers, immutable blocks) but also pointers to memory blocks for records.
%In general, we basically have to compare graphs---which implies structural comparison may diverge.
%
%Accordingly, the specification of $v_1\ \texttt{=}\ v_2$ requires the (partial) ownership of a \emph{memory footprint} corresponding to the union of the two compared graphs, giving the right to traverse them safely.
%If it terminates, the comparison decides whether the two graphs are isomorphic.
%In \Iris, this gives:
%
%\begin{minted}{coq}
%Axiom structeq_spec : ∀ `{zoo_G : !ZooG Σ} {v1 v2} footprint,
%  val_traversable footprint v1 →
%  val_traversable footprint v2 →
%  {{{ structeq_footprint footprint }}}
%    v1 = v2
%  {{{ b, RET #b;
%    structeq_footprint footprint ∗
%    ⌜ if b then val_structeq footprint v1 v2
%      else val_structne footprint v1 v2 ⌝
%  }}}.
%\end{minted}
%
%Obviously, this general specification is not very convenient to work with.
%Fortunately, for abstract tree-like values, we get a much simpler variant:
%
%\begin{minted}{coq}
%Lemma structeq_spec_abstract `{zoo_G : !ZooG Σ} v1 v2 :
%  val_is_abstract v1 →
%  val_is_abstract v2 →
%  {{{ True }}}
%    v1 = v2
%  {{{ RET #(bool_decide (v1 = v2)); True }}}
%Proof.
%  ...
%Qed.
%\end{minted}