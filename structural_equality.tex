\section{Structural equality}
\label{sec:structural_equality}

Structural equality is also supported.
More precisely, it is not part of the semantics of the language but axiomatized on top of it\footnote{We could also have implemented it in \ZooLang, but that would require more low-level primitives.}.
The reason is that it is in fact difficult to specify for arbitrary values.
In general, we have to compare graphs---which implies structural comparison may diverge.

Accordingly, the specification of $v_1\ \texttt{=}\ v_2$ requires the (partial) ownership of a \emph{memory footprint} corresponding to the union of the two compared graphs, giving the permission to traverse them safely.
If it terminates, the comparison decides whether the two graphs are isomorphic (modulo representation conflicts, as described in \cref{sec:physical_equality}).
In \Iris, this gives:

\begin{minted}{coq}
Axiom structeq_spec : ∀ `{zoo_G : !ZooG Σ} {v1 v2} footprint,
  val_traversable footprint v1 →
  val_traversable footprint v2 →
  {{{ structeq_footprint footprint }}}
    v1 = v2
  {{{ b, RET #b;
      structeq_footprint footprint ∗
      ⌜(if b then val_structeq else val_structneq) footprint v1 v2⌝
  }}}.
\end{minted}

Obviously, this general specification is not very convenient to work with.
Fortunately, for abstract values (without any mutable part), we can prove a much simpler variant saying that structural equality boils down to physical equality:

\begin{minted}{coq}
Lemma structeq_spec_abstract `{zoo_G : !ZooG Σ} v1 v2 :
  val_abstract v1 →
  val_abstract v2 →
  {{{ True }}}
    v1 = v2
  {{{ b, RET #b; ⌜(if b then val_physeq else val_physneq) v1 v2⌝ }}}
Proof. ... Qed.
\end{minted}
