The release of \OCaml~5, which introduced parallelism into the language, drove the need for safe and efficient concurrent data structures.
New libraries like \Saturn% ~\cite{saturn}
aim at addressing this need.
From the perspective of formal verification, this is an opportunity to apply and further state-of-the-art techniques to provide stronger guarantees.

We present a framework for verifying fine-grained concurrent \OCaml~5 algorithms.
We followed a pragmatic approach, studying \OCaml code written by concurrency expert to delimit a limited but sufficient fragment of the language to express those algorithms; the outcome is a dialect of \OCaml that we call \ZooLang.
We formalized its semantics carefully via a deep embedding in the \Rocq proof assistant.
We provide a tool to translate source \OCaml programs into \ZooLang syntax inside \Rocq, where they can be specified and verified using the \Iris% ~\cite{DBLP:journals/jfp/JungKJBBD18}
concurrent separation logic.

We verified fine-grained concurrent algorithms, along with subsets of the \OCaml standard library necessary to express them: the classic Treiber stack, and a use of reference-counting for file descriptors within the \Eio library.
This formalization work uncovered delicate questions of programming-language semantics, around physical equality for example.
In the process, we also extended \OCaml to more efficiently express certain concurrent programs.
