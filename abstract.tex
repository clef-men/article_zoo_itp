The release of \OCaml~5, which introduced parallelism into the language, drove the need for safe and efficient concurrent data structures.
New libraries like \Saturn aim at addressing this need.
From the perspective of formal verification, this is an opportunity to apply and further state-of-the-art techniques to provide stronger guarantees.

We present \Zoo, a framework for verifying fine-grained concurrent \OCaml~5 algorithms.
We followed a pragmatic approach, studying \OCaml code written by concurrency experts to delimit a limited but sufficient fragment of the language to express these algorithms: \ZooLang.
We formalized its semantics carefully via a deep embedding in the \Rocq proof assistant.
We provide a tool to translate source \OCaml programs into \ZooLang syntax inside \Rocq, where they can be specified and verified using the \Iris concurrent separation logic.

We verified a subset of the standard library along with fine-grained concurrent algorithms, including Treiber stack and a use of reference-counting for file descriptors from the \Eio library.
This formalization work uncovered delicate questions of programming language semantics, especially around physical equality.
In the process, we also extended \OCaml to more efficiently express certain concurrent programs.
